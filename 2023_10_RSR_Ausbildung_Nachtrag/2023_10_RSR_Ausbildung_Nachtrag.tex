\documentclass[a4paper,ngerman]{tui-algo-seminar}
\usepackage{graphicx}
\usepackage{algorithm2e}
\usepackage{booktabs}
\usepackage{tikz}
\usepackage{hyperref}
\usepackage{float}
\usepackage{listings}
\usepackage{totpages}  % Paket für die Gesamtseitenzahl
\setlength{\footskip}{13.0pt} % Fußzeilenabstand

\usepackage[utf8]{inputenc} % Verwenden Sie utf8 für UTF-8 Zeichencodierung

\newcommand{\inhalt}{Nachtrag zur RSR-Ausbildung 2023}
\seminar{\inhalt}
\semester{\today}
\title{\inhalt}
\author{Erik Skopp}
\nolinenumbers
\usepackage{fancyhdr}  % Für individuelle Kopf- und Fußzeilen

% Definieren des Seitenstils
\pagestyle{fancy}
\fancyhf{}  % Vorherige Kopf- und Fußzeilen-Einstellungen löschen
\fancyfoot[Rf]{\thepage ~von \zpageref{LastPage}}  % Gesamtseitenzahl

\begin{document}
\nolinenumbers
\maketitle
\thispagestyle{plain} % Seitenzahl auf Seite 1 anzeigen
\begin{abstract}
Bericht: \inhalt
\end{abstract}

\section{Nachtrag}
Am 17. September dieses Jahres haben die beiden Ilmenauer Schachfreunde Markus Eisenbach und Erik Skopp die Schiedsrichterprüfung des Thüringer Schachbundes abgelegt. Wer diese besteht, erhält den Titel "Regionaler Schiedsrichter im Schachsport" sowie den FIDE-Titel FA (FIDE Arbiter). Letzterer ist die unterste Ebene der FIDE-Schiedsrichter.\\
Während des Magdeburger Opens erreichte uns eine E-Mail vom Referat Ausbildung des Thüringer Schachbundes, verfasst von Norbert Reichel. Darin wurde schriftlich mitgeteilt, dass sowohl Erik Skopp als auch Markus Eisenbach die schriftliche Prüfung zum regionalen Schiedsrichter bestanden haben. Schachfreund Markus Eisenbach erzielte von möglichen 100 Punkten 93,5 Punkte. Schachfreund Erik Skopp erzielte ebenfalls von 100 möglichen Punkten 95 Punkte. Damit haben alle Teilnehmer vom Ilmenauer SV die Prüfung erfolgreich absolviert. Der beste Teilnehmer war Florian Jung vom SV Schott Jena mit 100 von 100 Punkten. Damit haben fast alle 19 Teilnehmer bestanden. Leider hat ein Schachfreund die Prüfung nicht bestanden.\\
Alle 18 Schachspieler haben nun beide Titel erhalten und bekommen ihre Schiedsrichterausweise aus Berlin zugeschickt. Damit sind sie offizielle Schiedsrichter.\\
\\
Erik Skopp
\end{document}
