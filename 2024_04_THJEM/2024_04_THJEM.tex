\documentclass[a4paper,ngerman]{tui-algo-seminar}
\usepackage{graphicx}
\usepackage{algorithm2e}
\usepackage{booktabs}
\usepackage{tikz}
\usepackage{hyperref}
\usepackage{float}
\usepackage{amsmath}
\usepackage{listings}
\usepackage{totpages}
%%\usepackage[margin=1in]{geometry}
\setlength{\footskip}{13.0pt}
\usepackage{thm-restate}

\usepackage[utf8]{inputenc}

\newcommand{\inhalt}{Thüringer Jugend Einzelmeisterschaft 2024}
\seminar{\inhalt}
\semester{\today}
\title{\inhalt}
\author{Erik Skopp}

\usepackage{fancyhdr}
\pagestyle{fancy}
\fancyhf{}
\nolinenumbers

\begin{document}

\maketitle
\thispagestyle{plain}
\begin{abstract}
    Bericht: \inhalt.\\
    Vom 4. bis zum 7. April fand in Naumburg (Sachsen-Anhalt) die diesjährige Thüringer Jugend-Einzelmeisterschaft im Schach statt. In jeder Altersklasse wurden sieben Runden nach dem Schweizer System ausgetragen. Die Gewinner qualifizieren sich für die Deutsche Einzelmeisterschaft in Willingen.
\end{abstract}

\tableofcontents 
\clearpage

\section{Bericht}

\section{Tabellen}
Alle Tabellen entstammen der Veröffentlichung der Thüringer Schachjugend unter folgender URL \url{https://ed.thsj.de/index.php/them-2024}\footnote{Der LInk wurde am \today ~abgerufen.}

% Auch wenn die Pfade nicht stimmen findet Overleaf die Projekte. Bei GitHub und dem Worfkflow ist das nicht so,
\subsection{Hanna Görlach}
    \subsubsection{Partien}
        \begin{table}[htbp]
\centering
\caption{Turnier Rangliste}
\begin{tabular}{|l|c|l|l|c|c|}
\hline
\multicolumn{6}{|c|}{Partien} \\
\hline
Runde & Farbe & Spieler & Verein & ELO & Ergebnis \\
\hline
1 & W & Grube, Anna (2.5) & ESV Lok Meiningen & 841 & 1 \\
2 & S & Brauer, Celiene (4) & SC Turm Erfurt & 1172 & 0.5 \\
3 & S & Ignatova, Gabriela (4.5) & Meuselwitzer SV & 1464 & 0.5 \\
4 & W & Richter, Amélie Elsa (4) & SG Blau-Weiß Stadtilm & 1205 & 0 \\
5 & W & Huth, Fabienne (4) & VfL 1990 Gera & 1072 & 0.5 \\
6 & S & Zeughardt, Selma (3) & Erfurter SK & 1006 & 0.5 \\
7 & W & Scheiding, Sophia (4.5) & Meuselwitzer SV & 1470 & 0 \\
\hline
\multicolumn{4}{|r|}{Gesamt} & Ø 1175 & 3.0 / 5, 13. Platz \\
\hline
\end{tabular}
\end{table}

\section{Bilder}
Bitte keine Bilder von Hanna auf die Website des Ilmenauer SV's\footnote{\url{https:ilmenauer-schachverein.de}} hochladen.
\vspace{0.5cm}
Alle Bilder finden Sie in der Cloud. Bitte nutzen Sie diese. Wir haben von Norbert Reichel via E-Mail die Rechte die Bilder für die Berichte zu nutzen. \\
Die Berechtigung liegt ab dem 19.04 dem Vorstand des Ilmenauer Schachvereines (Markus Hartung) vor.f
    \subsubsection{Rangliste U14w-U18w}
        \begin{table}[H]
\centering
\begin{tabular}{|c|l|l|l|c|c|c|c|c|c|}
\hline
Nr. & Titel & Spieler & Verein & TWZ & Sp & Pkt & BhZl-1 & SoBe-1 & + \\ \hline
1 & 16w & Eichhorn, Mathilda & SV Schott Jena & 1737 & 7 & 5.5 & 27.5 & 21.25 & 4 \\
2 & 18w & Ulrich, Helena Irene & SV Medizin Erfurt & 1732 & 7 & 5.5 & 27.0 & 19.75 & 4 \\
3 & 16w & Loos, Tilda & VfL 1990 Gera & 1500 & 7 & 5.0 & 26.5 & 17.25 & 3 \\
4 & 14w & Ignatova, Gabriela & Meuselwitzer SV & 1464 & 7 & 4.5 & 26.0 & 15.25 & 2 \\
5 & 16w & Scheiding, Sophia & Meuselwitzer SV & 1470 & 7 & 4.5 & 24.0 & 14.75 & 4 \\
6 & 16w & Niederdorfer, Mathilda & SV Empor Erfurt & 1257 & 7 & 4.0 & 27.0 & 14.00 & 3 \\
7 & 16w & Richter, Amélie Elsa & SG Blau-Weiß Stadtilm & 1205 & 7 & 4.0 & 24.5 & 12.00 & 4 \\
8 & 16w & Huth, Fabienne & VfL 1990 Gera & 1072 & 7 & 4.0 & 20.5 & 9.50 & 3 \\
9 & 16w & Brauer, Celiene & SC Turm Erfurt & 1172 & 7 & 4.0 & 20.0 & 10.75 & 3 \\
10 & 16w & Wicklein, Isabella & fuß brothers Jena & 941 & 7 & 3.5 & 20.5 & 10.25 & 3 \\
11 & 14w & Fota, Heidi & SV Empor Erfurt & 917 & 7 & 3.5 & 20.0 & 7.50 & 3 \\
12 & 14w & Le, Thi An & 1. Eichsfelder SC & 1105 & 7 & 3.5 & 19.5 & 7.00 & 3 \\
13 & 18w & Görlach, Hanna & Ilmenauer SV & 1122 & 7 & 3.0 & 24.0 & 10.25 & 1 \\
14 & 14w & Zeughardt, Selma & Erfurter SK & 1006 & 7 & 3.0 & 22.5 & 9.50 & 2 \\
15 & 14w & Stadelmann, Henriette & SSV Vimaria 91 Weimar & 1041 & 7 & 3.0 & 22.0 & 7.75 & 2 \\
16 & 14w & Grube, Anna & ESV Lok Meiningen & 841 & 7 & 2.5 & 24.0 & 5.75 & 2 \\
17 & 14w & Dilcher, Maria & SV Empor Erfurt & 804 & 7 & 2.5 & 23.5 & 5.00 & 2 \\
18 & 16w & Yusibova, Shams & SV Empor Erfurt & 999 & 7 & 2.5 & 20.5 & 5.00 & 2 \\
19 & 14w & Yusibova, Banu & SV Empor Erfurt & 976 & 7 & 2.0 & 20.5 & 3.00 & 1 \\ \hline
\end{tabular}
\caption{Rangliste U14w-U18w}
\label{tab:Rangliste_U14w-U18w}
\end{table}
\clearpage

\subsection{Kashvi Ray}
    \subsubsection{Partien}
        \begin{table}
\begin{tabular}{|c|c|l|l|c|}
\hline
\textbf{Partien} & \textbf{Runde} & \textbf{Spieler} & \textbf{Verein} & \textbf{Punkte} \\ \hline
S & Runde 1 & Ulbrich, Cäcilia (4) & 1. Eichsfelder SC & 1 \\ \hline
S & Runde 2 & Sniegowski, Amalia (6) & Meuselwitzer SV & 0 \\ \hline
W & Runde 3 & Buntin, Amelia Bernadette (5) & ZSG Grün-Weiß Waltershausen & 0 \\ \hline
S & Runde 4 & Nawatzki, Elenor Viktoria (1) & SG Blau-Weiß Stadtilm & 1 \\ \hline
S & Runde 5 & Heß, Julia (3.5) & MTV 1876 Saalfeld & 0 \\ \hline
W & Runde 6 & Nöthlich, Fenja (0.5) & SV Empor Erfurt & 0.5 \\ \hline
S & Runde 7 & Fischer, Emily (2.5) & SV Springer Oldisleben & 0.5 \\ \hline
\multicolumn{4}{|r|}{\textbf{Gesamt (3 Spieler)}} & \textbf{3.0/4} \\ \hline
\multicolumn{4}{|r|}{\textbf{Durchschnitt}} & \textbf{861} \\ \hline
\multicolumn{4}{|r|}{\textbf{Platz}} & \textbf{9.} \\ \hline
\end{tabular}
\caption{Partien Kashvi}
\label{label:Tabelle_Kashvi}
\end{table}
    \subsubsection{Rangliste U10w}
        \begin{table}[H]
\centering
\begin{tabular}{|c|l|l|c|c|c|c|c|c|}
\hline
Nr. & Spieler & Verein & TWZ & Sp & Pkt & BhZl-1 & SoBe-1 & + \\ \hline
1 & Sniegowski, Amalia & Meuselwitzer SV & 975 & 7 & 6.0 & 24.5 & 22.50 & 6 \\
2 & Buntin, Amelia Bernadette & ZSG Grün-Weiß Waltershausen & 887 & 7 & 5.0 & 26.0 & 18.50 & 5 \\
3 & Fritzsche, Emma & SV Empor Erfurt & - & 7 & 4.5 & 26.5 & 15.75 & 3 \\
4 & Eßers, Mathilda Marie & SG Blau-Weiß Stadtilm & 924 & 7 & 4.5 & 24.0 & 13.25 & 4 \\
5 & Möller, Lia Sofie & SG Blau-Weiß Stadtilm & 733 & 7 & 4.0 & 26.0 & 14.50 & 4 \\
6 & Ulbrich, Cäcilia & 1. Eichsfelder SC & - & 7 & 4.0 & 19.0 & 7.50 & 4 \\
7 & Beier, Viktoria & SV Empor Erfurt & - & 7 & 3.5 & 24.5 & 7.75 & 3 \\
8 & Heß, Julia & MTV 1876 Saalfeld & 723 & 7 & 3.5 & 23.5 & 8.25 & 3 \\
9 & Ray, Kashvi & Ilmenauer SV & 758 & 7 & 3.0 & 22.0 & 6.50 & 2 \\
10 & Fischer, Emily & SV Springer Oldisleben & - & 7 & 2.5 & 21.5 & 3.00 & 2 \\
11 & Nawatzki, Elenor Viktoria & SG Blau-Weiß Stadtilm & - & 7 & 1.0 & 22.0 & 0.50 & 1 \\
12 & Nöthlich, Fenja & SV Empor Erfurt & - & 7 & 0.5 & 21.0 & 1.50 & 0 \\ \hline
\end{tabular}
\caption{Rangliste U10w}
\label{tab:Rangliste_U10w}
\end{table}
\clearpage

\subsection{Ronika Nasiri}
    \subsubsection{Partien}
        \begin{table}[htbp]
\centering
\caption{Turnier Rangliste U12}
\begin{tabular}{|l|c|p{1.8in}|l|c|c|}
\hline
\multicolumn{6}{|c|}{Partien} \\
\hline
\textbf{Runde} & \textbf{Farbe} & \textbf{Spieler} & \textbf{Verein} & \textbf{ELO} & \textbf{Ergebnis} \\
\hline
1 & S & Ulbrich, Eusebia (4.5) & 1. Eichsfelder SC & 810 & 0 \\
2 & S & Weichel, Juliette (3.5) & SV Empor Erfurt & 778 & 0 \\
3 & W & Kührt, Daria (3.5) & SV Empor Erfurt & --- & 1 \\
4 & W & Böttcher, Paulina (2) & SV Springer Oldisleben & 788 & 0 \\
5 & S & Lehmann, Clara Theres (3) & SV Empor Erfurt & 935 & 0 \\
6 & W & Liebaug, Ellen (1.5) & SC Rochade Steinbach-Hallenberg & --- & 1 \\
7 & S & Hoyer, Clara (4) & SV Empor Erfurt & 798 & 0 \\
\hline
\multicolumn{4}{|r|}{Gesamt} & Ø 821 (5 Spieler) & 2.0 / 2, 12. Platz \\
\hline
\end{tabular}
\end{table}
    \subsubsection{Rangliste U12w}
        \begin{table}[H]
\centering
\begin{tabular}{|c|l|l|c|c|c|c|c|c|}
\hline
Nr. & Spieler & Verein & TWZ & Sp & Pkt & BhZl-1 & SoBe-1 & + \\ \hline
1 & Schille, Marlene & VfL 1990 Gera & 1125 & 7 & 6.0 & 25.5 & 20.50 & 5 \\
2 & Albrecht, Sidney Jenna & SV Empor Erfurt & 1295 & 7 & 5.5 & 26.5 & 21.50 & 5 \\
3 & Sniegowski, Victoria & Meuselwitzer SV & 1019 & 7 & 5.0 & 24.5 & 18.50 & 5 \\
4 & Ulbrich, Eusebia & 1. Eichsfelder SC & 810 & 7 & 4.5 & 27.5 & 16.00 & 4 \\
5 & Weißleder, Emia & SV Empor Erfurt & 1086 & 7 & 4.0 & 24.5 & 10.00 & 4 \\
6 & Hoyer, Clara & SV Empor Erfurt & 798 & 7 & 4.0 & 22.5 & 10.25 & 3 \\
7 & Weichel, Juliette & SV Empor Erfurt & 778 & 7 & 3.5 & 23.5 & 9.00 & 3 \\
8 & Kührt, Daria & SV Empor Erfurt & - & 7 & 3.5 & 17.0 & 6.50 & 3 \\
9 & Lehmann, Clara Theres & SV Empor Erfurt & 935 & 7 & 3.0 & 22.5 & 5.00 & 3 \\
10 & Gallerach, Elisa & ESV Lok Sömmerda & 916 & 7 & 3.0 & 21.5 & 6.75 & 2 \\
11 & Böttcher, Paulina & SV Springer Oldisleben & 788 & 7 & 2.0 & 22.0 & 5.00 & 2 \\
12 & Nasiri, Ronika & Ilmenauer SV & - & 7 & 2.0 & 20.5 & 5.00 & 2 \\
13 & Burkhardt, Charlotte & SC Rochade Steinbach-Hallenberg & - & 7 & 1.5 & 22.5 & 5.75 & 1 \\
14 & Liebaug, Ellen & SC Rochade Steinbach-Hallenberg & - & 7 & 1.5 & 18.5 & 2.75 & 1 \\ \hline
\end{tabular}
\caption{Turniertabelle}
\label{tab:turniertabelle}
\end{table}

\clearpage



\section{Anmerkungen}




Die Ergebnisse der THJEM 2024 sind, ...
% TODO
\end{document}
