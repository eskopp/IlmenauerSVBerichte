% Alle Optionen, die an `tui-algo-seminar` übergeben werden, werden an lipics-v2021 weitergereicht.
% Lesen Sie die Dokumentation für lipics-v2021 für weitere Informationen:
% https://submission.dagstuhl.de/series/details/5#author
\documentclass[a4paper,ngerman]{tui-algo-seminar}

% Nach Belieben (de-)kommentieren Sie die folgenden Zeilen:
\usepackage{graphicx}  % Grafiken einbinden
\usepackage{algorithm2e}  % Pseudocode-Listings
\usepackage{booktabs}  % ansprechendere Tabellen
\usepackage{tikz}  % Abbildungen zeichnen
\usepackage{hyperref} % URL

\newcommand{\content}{1. Ilmenauer Halloween Blitz 2023}
\seminar{\content}

\semester{13.11.2023}
\title{\content}
\author{Erik Skopp}

% Kommentieren Sie die folgende Zeile aus, um die Zeilennummern zu entfernen.
% Zeilennummern müssen in jeder Vorabversion enthalten sein.
% \nolinenumbers  % Diese Anweisung wird vom Paket `lineno` definiert

\begin{document}
	
	\maketitle
\thispagestyle{plain} % Seitenzahl auf Seite 1 anzeigen

\begin{abstract}


. 
\end{abstract}

\section{Bericht}
Am 03. November fand im Ilmenauer Spiellokal (EAZ) das erste Halloween-Blitzturnier statt. Gespielt wurden 11 Runden im Schweizer System mit einer Bedenkzeit von 5 Minuten ohne Inkrement. Jeder Schachspieler, der mitspielen wollte, war eingeladen, unabhängig davon, ob er Anfänger oder Profi war. Insgesamt nahmen 21 Schachspieler aus 6 Nationen teil. Neben Deutschland waren auch Nationen wie die Ukraine, Palästina, Russland und Indien vertreten. Es handelte sich um ein internationales Turnier.

Während des Turniers wurde für das leibliche Wohl gesorgt. Die Spitzengruppe bestand aus Matthias Liedmann (Ilmenauer SV), Marco Geißhirt (SG Barchfeld/Breitungen) und Pascal Eichenauer (SG Würzburg). Die meisten konnten in den ersten Runden ihre Punkte einfahren. Nur Marco strauchelte am Anfang etwas und konnte aus seinen ersten 3 Partien nur 1,5 Punkte holen, sodass Pascal mit einem halben Punkt Vorsprung den ersten Platz besetzte. Leider konnte Ainur (Ilmenauer SV) erst zur dritten Runde kommen und musste daher etwas weiter hinten starten. Dennoch gewann er gekonnt gegen Timo (Ilmenauer SV).

Nach Runde 6 schlich sich Stefan Schenk mit soliden 4 von 6 Punkten auf Platz 3. Diese Position konnte er sich auch bis Runde 8 halten. Dort verlor er gegen Marco. Da zu diesem Zeitpunkt bereits 3 Spieler das Turnier verlassen hatten, war immer jemand spielfrei. Die Nachwuchsspieler des Ilmenauer Schachvereins hatten indes ebenfalls ihre Punkte gesammelt. Nach Runde 8 waren Hanna mit 3 Punkten und Norik ebenfalls mit 3 Punkten unterwegs. Dies ist in diesem Teilnehmerfeld sehr solide.

In Runde neun trafen sich zwei der Favoriten am ersten Brett. Marco spielte gegen Ainur. Pascal rutschte nach dem Remis gegen Ainur in Runde 7 auf das zweite Brett ab. Nach Runde 10 lag Marco auf Platz 1, gefolgt von Pascal auf Platz 2 und Matthias auf Platz 3. Ainur gewann und remisierte seine Partien, man merkte jedoch, dass ihm die zwei fehlenden Partien vom Anfang fehlten. In der letzten Runde gewann Leonid (SV Empor Erfurt) gegen Marco am ersten Brett, und Pascal gewann gegen Fares am Brett 2. Dadurch waren Marco und Pascal punktgleich. Der Buchholzunterschied von +0,5 gab am Ende Marco den ersten Platz und Pascal den zweiten. Auf Platz drei landete Matthias, gefolgt von Ainur.

Das Turnier war ein Erfolg. 22 Schachspieler aus aller Welt haben sich versammelt, um ihrer Leidenschaft zu frönen. Dies hat uns dazu bewogen, zum Nikolaus ein weiteres Blitzturnier zu veranstalten. Alle Schachspieler sind dazu herzlich eingeladen.

Erik Skopp

\section{Bilder}
Die Bilder finden Sie in der Cloud des Ilmenauer Schachvereines: 
\begin{itemize}
	\item[-]: \url{https://cloud.ilmenauer-schachverein.de}
\end{itemize}

\section{Chess-Results}
\begin{itemize}
	\item[-] https://chess-results.com/tnr839981.aspx?lan=1
\end{itemize}
\clearpage

\section{Tabellen}
	\begin{center}
	\begin{tabular}{|c|c|l|c|c|c|c|c|c|c|}
		\hline
		\textbf{Rk.} & \textbf{SNo} & \textbf{Name} & \textbf{FED} & \textbf{Rtg} & \textbf{Club/City} & \textbf{Pts.} & \textbf{TB1} & \textbf{TB2} & \textbf{TB3} \\
		\hline
		1 & 3 & Geißhirt, Marco & GER & 1998 & SG Barchfeld/Breitungen & 8,5 & 70 & 54,00 & 8 \\
		2 & 1 & Eichenauer, Pascal & GER & 2097 & SV Würzburg von 1865 e.V. & 8,5 & 67 & 47,25 & 8 \\
		3 & 2 & Liedmann, Matthias & GER & 2050 & Ilmenauer SV & 7,5 & 73,5 & 45,00 & 7 \\
		4 & 22 & Ziganshin, Ainur & RUS & 2198 & Ilmenauer SV & 7,5 & 66,5 & 42,25 & 7 \\
		5 & 5 & Elkhanov, Leonid & GER & 1813 & SV Empor Erfurt & 7 & 73 & 42,50 & 7 \\
		6 & 8 & Dudeja, Iresh & IND & 1586 & Ilmenauer SV & 7 & 70 & 42,75 & 6 \\
		7 & 9 & Hartung, Markus & GER & 1584 & Ilmenauer SV & 6 & 71,5 & 31,75 & 5 \\
		8 & 4 & Schenk, Stefan & GER & 1909 & Ilmenauer SV & 6 & 68 & 30,50 & 6 \\
		9 & 12 & Eisenbach, Markus Dr. & GER & 1404 & Ilmenauer SV & 6 & 55,5 & 21,50 & 6 \\
		10 & 7 & Lehmann, Georg & GER & 1586 & ESV Lok Meiningen & 6 & 45,5 & 20,50 & 6 \\
		11 & 10 & Skopp, Erik & GER & 1561 & Ilmenauer SV & 5,5 & 67,5 & 29,75 & 5 \\
		12 & 21 & Schvachko, Mark & UKR & 0 & & 5,5 & 59 & 22,25 & 5 \\
		13 & 15 & Abuawad, Fares & PSE & 0 & Ilmenauer SV & 5 & 65 & 22,00 & 5 \\
		14 & 19 & Jung, Timo & GER & 0 & & 5 & 56 & 18,50 & 5 \\
		15 & 13 & Lehmann, Norik & GER & 886 & Ilmenauer SV & 5 & 51,5 & 14,00 & 5 \\
		16 & 6 & Michael, Torsten & GER & 1680 & Ilmenauer SV & 4 & 57,5 & 13,50 & 4 \\
		17 & 17 & Görlach, Hanna Pauline & GER & 0 & Ilmenauer SV & 4 & 46,5 & 13,50 & 4 \\
		18 & 11 & Böhmer, Leon & GER & 974 & Ilmenauer SV & 3 & 54 & 8,00 & 3 \\
		19 & 14 & Winger, Frank & GER & 838 & Ilmenauer SV & 2 & 51,5 & 6,00 & 2 \\
		20 & 18 & Greul, Simon & GER & 0 & Ilmenauer SV & 1 & 48 & 5,50 & 1 \\
		21 & 20 & Möller, Nico & GER & 0 & & 1 & 40,5 & 4,00 & 1 \\
		\hline
	\end{tabular}
\end{center}

\end{document}

