\documentclass[a4paper,ngerman]{tui-algo-seminar}
\usepackage{graphicx}
\usepackage{algorithm2e}
\usepackage{booktabs}
\usepackage{tikz}
\usepackage{hyperref}
\usepackage{float}
\usepackage{listings}
\usepackage{totpages}  % Paket für die Gesamtseitenzahl
\setlength{\footskip}{13.0pt} % keine Ahnung ^^

\usepackage[utf8]{inputenc} % Verwenden Sie utf8 für UTF-8 Zeichencodierung

\newcommand{\inhalt}{28. Magdeburger A-Open 2023}
\seminar{\inhalt}
\semester{\today}
\title{\inhalt}
\author{Erik Skopp}

\usepackage{fancyhdr}  % Für individuelle Kopf- und Fußzeilen
\nolinenumbers

% Definieren des Seitenstils
\pagestyle{fancy}
\fancyhf{}  % Vorherige Kopf- und Fußzeilen-Einstellungen löschen
\fancyfoot[Rf]{\thepage ~von \zpageref{LastPage}}  % Gesamtseitenzahl

\begin{document}

\maketitle
\thispagestyle{plain} % Seitenzahl auf Seite 1 anzeigen
\begin{abstract}
Bericht: \inhalt.\\
Das 
\end{abstract}

\section{Bericht}
Hier können Sie eine Einführung oder den Haupttext Ihres Berichts einfügen.

\section{Links}
\begin{itemize}
    \item[-] asd
    \item[-] as
\end{itemize}

\section{Anhang}
\subsection{PGN-Dateien}
In der Cloud liegen 3 PGN-Dateien vor:
\begin{itemize}
    \item Platzhalter für PGN-Datei 1
    \item Platzhalter für PGN-Datei 2
    \item Platzhalter für PGN-Datei 3
\end{itemize}

\subsection{Bilder}
Hier können Sie Beschreibungen für die Bilder einfügen, die Sie im Anhang haben.

\end{document}
